\documentclass[12pt,a4paper]{article}
% Pri pouziti masivnejsich fontu (pripad pana Jurenky) lze uzit mensi pismo:
%\documentclass[11pt,a4paper]{article}
%
\usepackage{color} %pro barevné odkazy, pøíp. nadpisy
\definecolor{odkazy}{rgb}{0.21,0.27,0.53} %tmavì modrá
\definecolor{nadpisy}{rgb}{0.5812,0.0665,0.0659} %cihlová
%
% Volba pdf driveru:
%\usepackage[dvipdfm]{hyperref}%
%\usepackage[pdftex]{hyperref}%
% Parametry prevodu do pdf
\providecommand{\hypersetup}[1]{}%
\hypersetup{%
%unicode,% ? Pravdepodobne bezvyznamne
pdfauthor={jméno autora},
pdftitle={název práce},
pdfsubject={subject práce},
pdfkeywords={klicova slova},
pdffitwindow=false,% Inicialni umisteni textu v okne Readeru
bookmarksopen=true,% Panel zalozek inicialne zobrazen
% Je-li tohle nastaveno jinak, nektere odkazy nekdy nefunguji
hypertexnames=false,
plainpages=false,
%pdfpagelabels,
%
breaklinks=true,% Radkovy lom smi prijit do klikatelneho odkazu
linkcolor=odkazy,% Graficka podoba odkazu
citecolor=odkazy,% ...
colorlinks=true,% ...
pdfhighlight=/O% ... (vzhled odkazu pri stisknuti)
}%
% Inputenc je asi zbytecne.
% Alespon v MikTeXu je kodovani lepsi specifikovat v prikazovem radku:
%\usepackage[cp1250]{inputenc}
% Option 'split' ovlivnuje deleni slov obsahujicich v sobe rozdelovnik
\usepackage[czech]{babel} %dnes už je však hotová integrace èeštiny do babelu
%\usepackage[split]{czech} %dnes už je však hotová integrace èeštiny do babelu
%
\usepackage{logdp} %užiteèné drobnosti
\usepackage{amsthm} %lepší práce s vìtami
\usepackage{amsmath} %nová prostøedí pro matematiku a vylepšení tìch stávajících
\usepackage{latexsym,amsfonts,amssymb} %nová písmenka
\usepackage{fancyhdr} %záhlaví a zápatí
\usepackage[nottoc]{tocbibind} %pøidá do obsahu položky Literatura a Rejstøík
%\usepackage{colorsection} %chcete-li barevné nadpisy,
% (použije se barva z definice \definecolor{nadpisy})
% Makra pro sazbu dukazu v gentzenovskem kalkulu vytvoril Sam Buss
%\usepackage{bussproofs}

% Zmena radkovani je uzitecna pri volbe masivnich fontu, viz vyse
%\def\baselinestretch{1.2} %øádkování
\pagestyle{plain}
%pøedbìžné nastavení hlavièky (balík fancyhdr)
%\headheight 13.6pt %možná ji bude tøeba zvednout, fancyhdr si pak stìžuje: \headheight
% too small, make it at least Xpt
\headheight 14.5pt %možná ji bude tøeba zvednout, fancyhdr si pak stìžuje: \headheight too \fancyhead{}
\fancyhead[R]{\leftmark}
\fancyfoot{}
\fancyfoot[C]{\thepage}
%
\newtheorem{veta}{Vìta}%[subsection]
\newtheorem{lemma}{Lemma}
\newtheorem{tvrzeni}{Tvrzení}
%a neèíslované varianty
\newtheorem*{vetax}{Vìta}
\newtheorem*{lemmax}{Lemma}
\newtheorem*{tvrzenix}{Tvrzení}
\theoremstyle{definition}
\newtheorem*{definice}{Definice}
%\usepackage{makeidx}
%\makeindex %bude-li rejstøík
\begin{document}
%titulní stránka
\begin{titlepage}
%\fontsize{16.16pt}{25pt}\selectfont
\Large
\begin{center}
Univerzita Karlova v Praze, Filozofická fakulta\\
Katedra logiky

\vspace{8.5em}
\textsc{Michal Ketner}\\[1.4em]
{%\color{nadpisy} %pro pøíp. barevný název
Z\'{a}po\v{c}tov\'{a} cvi\v{c}en\'{i} a probl\'{e}my\\[6.8em]
2015}
\end{center}
\end{titlepage}\

\tableofcontents
\clearpage







\pagestyle{fancy} %detailní definice chování záhlaví
\renewcommand{\sectionmark}[1]{\markboth{\slshape\thesection.\ #1}{}}
\section{Cvi\v{c}en\'{i} 1 }
\subsection{V\v{e}ta a}
Vzpom\'{i}nka na tuto pohnutou ud\'{a}lost a osobn\'{i}  za\v{z}itek z jedine\v{c}n\'{e}ho rozhledu z vrcholku hory T\'{a}bor podn\'{i}tily patrn\v{e} obyvatele k tomu, \v{z}e dali po\v{r}\'{i}dit mapu zasl\'{i}ben\'{e}  zem\v{e} s jej\'{i}mi posv\'{a}tn\'{y}mi m\'{i}sty a z t\'{e}to ilustrace k Bibli si ud\v{e}lali zvl\'{a}\v{s}tn\'{i}  v\'{y}zdobu sv\'{e}  svatyn\v{e}.

\textbf{Morfologick\'{a}} \\
\begin{tabular}{|c||c|}
\hline
Na  $ \rightarrow $ ud\'{a}lost &  mapu $ \rightarrow  $ jej\'{i}mi \\
\hline 
Na  $ \rightarrow $ tuto  & s $ \rightarrow  $ jej\'{i}mi  \\
\hline 
Na  $ \rightarrow $ pohnutou &  s $ \rightarrow  $ posv\'{a}tnymi\\
\hline 
ud\'{a}lost  $ \rightarrow $ tuto & s $ \rightarrow  $ m\'{i}sty \\
\hline 
ud\'{a}lost  $ \rightarrow $ pohnutou & m\'{i}sty $ \rightarrow  $ jej\'{i}mi  \\
\hline
z\'{a}\v{z}itek $ \rightarrow $ osobn\'{i} & m\'{i}sty $ \rightarrow  $ posv\'{a}tnymi \\
\hline
Z  $ \rightarrow $ jedine\v{c}n\'{e}ho & z $ \rightarrow  $ ilustrace \\
\hline
Z  $ \rightarrow $ rozhledu & z $ \rightarrow  $ t\'{e}to \\
\hline 
Rozhledu  $ \rightarrow $ jedine\v{c}n\'{e}ho & k $ \rightarrow  $ Bibli \\
\hline
Z  $ \rightarrow $ vrcholku &  obyvatel\'{e} $ \rightarrow  $ ud\v{e}lali  \\
\hline
vrcholku  $ \rightarrow $ hory & ud\v{e}lali $ \rightarrow  $ v\'{y}zdobu \\
\hline
hory  $ \rightarrow $ T\'{a}bor & ud\v{e}lali $ \rightarrow  $ zvl\'{a}\v{s}tn\'{i} \\
\hline
vzpom\'{i}nka a z\'{a}\v{z}itek $ \rightarrow $ podn\'{i}tily & v\'{y}zdobu $ \rightarrow  $ zvl\'{a}\v{s}tn\'{i} \\
\hline
podn\'{i}tily  $ \rightarrow $ obyvatele &  v\'{y}zdobu $ \rightarrow  $ sv\'{e} \\
\hline
k  $ \rightarrow $ tomu & v\'{y}zdobu $ \rightarrow  $ svatyn\v{e}  \\
\hline
podn\'{i}tily  $ \rightarrow $ tomu & svatyn\v{e} $ \rightarrow  $ sv\'{e}  \\
\hline
obyvatele $ \rightarrow $ dali & mapu $ \rightarrow  $ zasl\'{i}ben\'{e} \\
\hline
dali $ \rightarrow $ po\v{r}\'{i}dit & mapu $ \rightarrow  $ zem\v{e} \\
\hline
po\v{r}\'{i}dit $ \rightarrow mapu $ & m\'{i}sty $ \rightarrow  $ posv\'{a}tnymi  \\
\hline
zem\v{e} $ \rightarrow  $ zasl\'{i}ben\'{e} & m\'{i}sty $ \rightarrow  $ jej\'{i}mi  \\
\hline

\end{tabular}

\clearpage
\textbf{Syntaktick\'{a}} \\
\begin{tabular}{|c||c|}
\hline
podn\'{i}tily$ \rightarrow $vzpom\'{i}nka a z\'{a}\v{z}itek & k tomu $ \rightarrow $ \v{z}e dali po\v{r}\'{i}dit a ud\v{e}lali  \\
\hline
z\'{a}\v{z}itek $ \rightarrow $ osobn\'{i} & dali po\v{r}\'{i}dit  $ \rightarrow $ mapu \\
\hline
vzpom\'{i}nka $\rightarrow $ na  ud\'{a}lost &   mapu $ \rightarrow  $ zem\v{e} \\
\hline
ud\'{a}lost  $ \rightarrow $ tuto & zem\v{e} $ \rightarrow  $ zasl\'{i}ben\'{e}  \\
\hline
ud\'{a}lost  $ \rightarrow $ pohnutou & zem\v{e} $ \rightarrow  $ s m\'{i}sty  \\
\hline
z\'{a}\v{z}itek $ \rightarrow $ osobn\'{i} & podn\'{i}tily  $ \rightarrow $ patrn\v{e} \\
\hline
z\'{a}\v{z}itek   $ \rightarrow $ z rozhledu &  ud\v{e}lali $ \rightarrow $ z ilustrace\\
\hline
Rozhledu  $ \rightarrow $ jedine\v{c}n\'{e}ho  &  ilustrace $ \rightarrow $ t\'{e}to  \\
\hline
Rozhledu  $ \rightarrow $ z vrcholku &  ilustrace $ \rightarrow $ k Bibli  \\
\hline
vrcholku $ \rightarrow $ hory& ud\v{e}lali $ \rightarrow $ v\'{y}zdobu  \\
\hline
hory  $ \rightarrow $ T\'{a}bor  & v\'{y}zdobu $\rightarrow$ svatynv\'{e}  \\
\hline
podn\'{i}tily  $ \rightarrow $ obyvatele & svatyn\v{e} $\rightarrow$ sv\'{e}  \\
\hline
 podn\'{i}tily  $ \rightarrow $ k tomu & \\
 \hline
 \hline
\end{tabular} \\

\textbf{S\'{e}mantick\'{a}} \\
\begin{tabular}{|c||c|} 
\hline
Na  $ \rightarrow $ ud\'{a}lost & zasl\'{i}ben\'{e} $ \rightarrow $ zem\v{e} \\ 
\hline
pohnutou $ \rightarrow $ ud\'{a}lost & mapu $ \rightarrow $ zem\v{e}  \\ 
\hline
 tuto $ \rightarrow $ ud\'{a}lost & s $ \rightarrow $ m\'{i}sty  \\ 
\hline
 osobn\'{i}  $ \rightarrow $ za\v{z}itek &  jej\'{i}mi $ \rightarrow $ m\'{i}sty \\
\hline
 z $ \rightarrow $  rozhledu & zasl\'{i}ben\'{y}mi $ \rightarrow $ m\'{i}sty  \\
\hline
 jedine\'{c}n\'{e}ho $ \rightarrow $  rozhledu  & z $ \rightarrow $ ilustrace \\
\hline
z $ \rightarrow $ vrcholku  & t\'{e}to $ \rightarrow $ ilustrace \\
\hline
podn\'{i}tily$ \rightarrow $vzpom\'{i}nka a z\'{a}\v{z}itek & k $ \rightarrow $ Bibli \\
\hline
 podn\'{i}tily  $ \rightarrow $ obyvatele & si ud\v{e}lali $ \rightarrow $ obyvatele \\
\hline
k $ \rightarrow $ tomu  & si ud\v{e}lali $ \rightarrow $ obyvatele \\
\hline
 dali po\v{r}\'{i}dit $ \rightarrow $ obyvatele  & zvl\'{a}\v{s}tn\'{i}$ \rightarrow $ v\'{y}zdobu\\
\hline
 dali po\v{r}\'{i}dit $ \rightarrow $ mapu & v\'{y}zdobu $ \rightarrow $ svatyn\v{e} \\
\hline
   & sv\'{e} $ \rightarrow $ svatyn\v{e} \\
\hline
\hline
\end{tabular}
\clearpage
\subsection{V\v{e}ta b}
Je to d\'{i}lo nepostradetln\'{e} pro v\v{s}echny proz\'{i}rav\'{e} a zv\'{i}dav\'{e} lidsk\'{e} mozky, v n\v{e}m\v{z} ka\v{z}d\'{y}, kdo miluje studium filosofie, perspektivy, malby, socha\v{r}stv\'{i}, architektury, hudby, a dal\v{s}\'{i}ch matematick\'{y}ch discipl\'{i}n, nalezne velmi jemn\'{e}, subtiln\'{i} a znamenit\'{e} pou\v{c}en\'{i} a tak\'{e} se pot\v{e}\v{s}\'{i} r\r{u}znorod\'{y}mi ot\'{a}zkami, je\v{z} se dot\'{y}kaj\'{i} velmi tajemn\'{y}ch v\v{e}domost\'{i}.

\textbf{Morfologick\'{a}} \\
\begin{tabular}{|c||c|}
\hline
d\'{i}lo $ \rightarrow $  nepostradetln\'{e} & studium $ \rightarrow  $  discipl\'{i}n \\
\hline
pro $ \rightarrow $  proz\'{i}rav\'{e} & studium $ \rightarrow  $  matematick\'{y}ch \\
\hline
pro $ \rightarrow $  zv\'{i}dav\'{e} & nalezne  $ \rightarrow $ jemn\'{e} \\
\hline
pro $ \rightarrow $  lidsk\'{e} & nalezne  $ \rightarrow $ subtiln\'{i} \\
\hline
pro $ \rightarrow $  mozky & nalezne  $ \rightarrow $ znamenit\'{e} \\
\hline
mozky $ \rightarrow $  proz\'{i}rav\'{e} & nalezne  $ \rightarrow $ pou\v{c}en\'{i} \\
\hline
mozky $ \rightarrow $  zv\'{i}dav\'{e} & ka\v{z}d\'{y}  $ \rightarrow $ se pot\v{e}\v{s}\'{i} \\
\hline
mozky $ \rightarrow $  lidsk\'{e} & se pot\v{e}\v{s}\'{i} $ \rightarrow $ r\r{u}znorod\'{y}mi \\
\hline
v $ \rightarrow $  n\v{e}m\v{z} & se pot\v{e}\v{s}\'{i} $ \rightarrow $ ot\'{a}zkami \\
\hline
ka\v{z}d\'{y}  $ \rightarrow $ nalezne & ot\'{a}zkami  $ \rightarrow $ r\r{u}znorod\'{y}mi \\
\hline
miluje $ \rightarrow $ studium & ot\'{a}zkami  $ \rightarrow $ je\v{z} \\
\hline
studium $ \rightarrow $ filosofie & dot\'{y}kaj\'{i}c\'{i} se $ \rightarrow $ v\v{e}domost\'{i} \\
\hline
studium $ \rightarrow $ perspektivy & dot\'{y}kaj\'{i}c\'{i} se $ \rightarrow $ tajemn\'{y}ch \\
\hline
studium $ \rightarrow $ perspektivy & v\v{e}domost\'{i}  $ \rightarrow $ tajemn\'{y}ch \\
\hline
studium $ \rightarrow $ malby & studium  $ \rightarrow $ socha\v{r}stv\v{i}\\
\hline
studium $ \rightarrow $ hudby & studium  $ \rightarrow $ dal\v{s}\'{i}ch \\
\hline

\end{tabular}

\clearpage
\textbf{Syntaktick\'{a}} \\
\begin{tabular}{|c||c|}
\hline
je nepostradateln\'{y} $ \rightarrow $ d\'{i}lo & je nepostradateln\'{y} $ \rightarrow $ pro mozky \\
\hline
pro mozky $ \rightarrow $ v\v{s}echny  & pro mozky $ \rightarrow $ proz\'{i}rav\'{e}  \\
\hline
pro mozky $ \rightarrow $ lidsk\'{e}  & pro mozky $ \rightarrow $ zv\'{i}dav\'{e}   \\
\hline
d\'{i}lo $ \rightarrow $ v n\v{e}m\v{z} ka\v{z}d\'{y} nalezne a pot\v{e}\v{s}\'{i} se & miluje $ \rightarrow $ studium \\
\hline
ka\v{z}d\'{y} $ \rightarrow $ kdo miluje & studium $ \rightarrow $ filosofie \\
\hline
studium $ \rightarrow $ perspektivy & studium $ \rightarrow $ malby \\
\hline
studium $ \rightarrow $ socha\v{r}stv\v{i} & studium $ \rightarrow $ architektury \\
\hline
studium $ \rightarrow $ hudby & studium $ \rightarrow $ discipl\'{i}n \\
\hline
discipl\'{i}n $\rightarrow $ matematick\'{y}ch  & discipl\'{i}n $\rightarrow $ dal\v{s}\'{i}ch \\
\hline
nalezne $\rightarrow $ ponau\v{c}en\'{i} & ponau\v{c}en\'{i} $\rightarrow $ jemn\'{e} \\
\hline
ponau\v{c}en\'{i} $\rightarrow $ znamenit\'{e} & ponau\v{c}en\'{i} $\rightarrow $ subtiln\'{i} \\
\hline
jemn\'{e} $ \rightarrow $ velmi & pot\v{e}\v{s}\'{i} se $ \rightarrow $ tak\'{e} \\
\hline
ot\'{a}zkami $ \rightarrow $ r\r{u}znorod\'{y}mi & pot\v{e}\v{s}\'{i} se $\rightarrow $ ot\'{a}zkami \\ 
\hline
ot\'{a}zkami $ \rightarrow $ je\v{z} se dot\'{y}kaj\'{i} & dot\'{y}kaj\'{i} se $\rightarrow $ v\v{e}domost\'{i} \\
\hline
v\v{e}domost\'{i} $ \rightarrow $ tajemn\'{y}ch & tajemn\'{y}ch $\rightarrow $  velmi \\
\hline
\hline
\end{tabular}
\textbf{Semantick\'{a}} \\
\begin{tabular}{|c||c|}
\hline
nepostradateln\'{e} $ \rightarrow $ d\'{i}lo & pro  $\rightarrow $ mozky \\
\hline
v\v{s}echny $ \rightarrow $ mozky & proz\'{i}rav\'{e} $ \rightarrow $ mozky \\
\hline
lidsk\'{e} $ \rightarrow $ mozky & zv\'{i}dav\'{e}  $\rightarrow $ mozky \\
\hline
matematick\'{y}ch $ \rightarrow $ discipl\'{i}n & dal\v{s}\'{i}ch $ \rightarrow $ discipl\'{i}n \\
\hline
nalezne $ \rightarrow $ pou\v{c}en\'{i} & jemn\'{e} $ \rightarrow $ pou\v{c}en\'{i} \\
\hline
subtiln\'{i} $ \rightarrow $ pou\v{c}en\'{i} & znamenit\'{e} $ \rightarrow $ pou\v{c}en\'{i} \\
\hline
nalezne $ \rightarrow $ ka\v{z}d\'{y} & se pot\v{e}\v{s}\'{i} $ \rightarrow $ ka\v{z}d\'{y} \\
\hline
r\r{u}znorod\'{y}mi $ \rightarrow $ ot\'{a}zkami & se pot\v{e}\v{s}\'{i} $ \rightarrow $ ot\'{a}zkami \\
\hline
dot\'{y}kaj\'{i} se $ \rightarrow $ ot\'{a}zkami & dot\'{y}kaj\'{i} se $ \rightarrow $ v\v{e}domosti \\
\hline
v $ \rightarrow $ n\v{e}m\v{z} & miluje $ \rightarrow $ ka\v{z}d\'{y} \\
\hline
miluje $ \rightarrow $ studium & \\

\end{tabular}
\clearpage
\section{Cvi\v{c}en\'{i} 2 }
z\'{a}pis bude rozd\v{e}len\'{y} na morf\'{e}my a dopln\v{e}n s\'{e}maty v z\'{a}vorce \\
p\'{a}dov\'{y} synkretismus je odd\v{e}len * \\
slovn\v{e}druhov\'{a} mnohozna\v{c}nost / \\
\subsection{V\v{e}ta a}
=Vzpom\'{i}n=k=a (substantivum, femin; sg ;1.p.) \\
=na=== (p\v{r}edlo\v{z}ka, se 4.p.*6.p.) \\
=t=uto (z\'{a}jmn\'{e}no ukazovac\'{i}, femin.; sg ; 4.p.) \\
=pohnut==ou= (adjectivum, femi.; sg; 4.p*7.p) \\
=ud\'{a}lost===(substantivum, fem; sg; 1.p.*4.p.) \\
=a===(spojka/ \v{c}\'{a}stice) \\
=osobn=\'{i}==(adjectivum, [mas. l ; [sg; 1.p.*4.p], [pl;1.p*.4.p*5.p]]; 
[mas n, [sg;1.p*.4.p*5.p],[pl;1.p*.4.p*5.p]] 
[femin, [sg;1-*7.p.];[pl;1.p*.4.p*5.p]] 
[neu., [[sg;1.p*.4.p*5.p],[pl;1.p*.4.p*5.p]] \\
=z\'{a}\v{z}it=ek==(substantivum, mas n ; sg; 1.p.*4.p)\\
=z===(p\v{r}edlo\v{z}ka, se 2.p)\\
=jedine\v{c}n==\'{e}ho=(adjectivum, mas.; sg; 2.p.*4.p.) \\
=rozhled==u=(substantivum, mas; sg; 2.p.*3.p*6.p)\\
=vrchol=k=u=(substantivum, mas; sg; 2.p.*3.p*5.p*6.p)\\
=ho=r=y=(substantivum, femin.[sg;2.p],[pl;1.p*4.p*5.p])\\
=T\'{a}bo=r==(vlast.jmen./substantivum, mas n; sg;1.p.*4.p.) \\
=podn=\'{i}til=y(verb, 3.os;pl; act;oznam) \\
=patrn\v{e}===(p\v{r}\'{i}slovce) \\
=obyvatel==e=(substantivum,[mas l.;[sg;2.p*4.p],[pl;4.p.]]) \\
=k===(p\v{r}edlo\v{z}ka) \\
=t=omu=(z\'{a}jmn\'{e}no ukazovac\'{i},[[neu.;sg;3.p],[mas. l.;sg;3.p],[mas. n.;sg;3.p]] \\
=\v{z}e===(spojka/ \v{c}\'{a}stice) \\
=d=al=i=(verb,3.os;pl;act;oznam;) \\
=po\v{r}=\'{i}di=t=(verb,infinitiv) \\
=map==u=(substantivum, fem,sg,4.p) \\
=zasl\'{i}ben==\'{e}=(adjectivum,[[fem;[sing,[2.p*3.p*6.p]],[pl,[2.p*3.p*5.p]]],[mas l;pl,4.p],[mas n;[pl,[2.p*3.p*5.p]]], [neu,[sg,[2.p*3.p*5.p]]]) \\
=zem==\v{e}=(substantivum,fem;[sg,[2.p*5.p]],[pl,[1.p.*4.p*5.p.]]) \\
=s===p\v{r}edlo\v{z}ka) \\
=jej\'{i}=mi==(z\'{a}jmn\'{e}no p\v{r}ivlas.,pl;7.p; v\v{s}echny rody) \\
=posv\'{a}tn==\'{y}mi=(adjectivum,pl;7.p; v\v{s}echny rody) \\
=m\'{i}st==y=(substantivum,neu;pl.;7.p./(p\v{r}\'{i}slovce) \\
=a===(spojka/ \v{c}\'{a}stice) \\
=z===(p\v{r}edlo\v{z}ka, se 2.p)\\
=t=\'{e}to==(z\'{a}jmn\'{e}no ukazovac\'{i},fem;sg.;2.p.*3.p.*6.p.) \\
=ilustrac==e=(substantivum,fem;[sg;1.p.*5.p],[pl;1.p.*4.p.*5.p.] )\\
=k===(p\v{r}edlo\v{z}ka se 3.p)\\
=Bibl==i=(vlast.jmen./substantivum;fem;sg.;3.p*4.p*6.p) \\
=s=i==(z\'{a}jmn\'{e}no  zvrat.) \\
=ud\v{e}l=al=i=(verb, 3.os,pl, ozn,acti) \\
=zvl\'{a}\v{s}t=n\'{i}==(adjectivum, [mas. l ; [sg; 1.p.*4.p], [pl;1.p*.4.p*5.p]]; 
[mas n, [sg;1.p*.4.p*5.p],[pl;1.p*.4.p*5.p]] 
[femin, [sg;1-*7.p.];[pl;1.p*.4.p*5.p]] 
[neu., [[sg;1.p*.4.p*5.p],[pl;1.p*.4.p*5.p]] )\\
=v\'{y}zdob==u=(substantivum,fem;sg.;4.p) \\
=sv=\'{e}==(z\'{a}jmn\'{e}no  p\v{r}ivlas. [fem;[sing,[2.p*3.p*6.p]],[pl,[2.p*3.p*5.p]]],[mas l;pl,4.p],[mas n;[pl,[2.p*3.p*5.p]]], [neu,[sg,[2.p*3.p*5.p]]] ) \\
svaty=n=\v{e}=(substantivum,fem;[sg,[1.p.*2p.*5.p]],[pl,[1.p*4.p*5.p]]) \\
\clearpage
\subsection{V\v{e}ta b}
==je==(verb, 3.os. sg, act, ozn/zajm. pl; 4.p.; vsech.rody,sg; 4.p.) \\
=t=o==(z\'{a}jmn\'{e}no ukazovac\'{i}, neu.;sg;1.p*4.p/\v{c}\'{a}stice) 
=d\'{i}l=o=(substantivum,neu;sg;1.p*4.p*5.p) \\
ne=postradteln==\'{e}=(adjectivum, [fem;[sing,[2.p*3.p*6.p]],[pl,[2.p*3.p*5.p]]],[mas l;pl,4.p],[mas n;[pl,[2.p*3.p*5.p]]], [neu,[sg,[2.p*3.p*5.p]]])\\
=pro===(p\v{r}edlo\v{z}ka, se 4.p)\\
=v\v{s}=echny==(z\'{a}jmeno,[mas. l.;pl;4p],[mas n.;pl;[1.p*4.p.]],[fem.;pl;[1.p*4.p.]] \\
=proz\'{i}rav==\'{e}=(adjectivum, [fem;[sing,[2.p*3.p*6.p]],[pl,[2.p*3.p*5.p]]],[mas l;pl,4.p],[mas n;[pl,[2.p*3.p*5.p]]], [neu,[sg,[2.p*3.p*5.p]]])\\
=a===(spojka/ \v{c}\'{a}stice) \\
=zv\'{i}dav==\'{e}=(adjectivum, [fem;[sing,[2.p*3.p*6.p]],[pl,[2.p*3.p*5.p]]],[mas l;pl,4.p],[mas n;[pl,[2.p*3.p*5.p]]], [neu,[sg,[2.p*3.p*5.p]]])\\
=lid=sk=\'{e}=(adjectivum, [fem;[sing,[2.p*3.p*6.p]],[pl,[2.p*3.p*5.p]]],[mas l;pl,4.p],[mas n;[pl,[2.p*3.p*5.p]]], [neu,[sg,[2.p*3.p*5.p]]])\\
=moz=k=y=(substantivum,mas;pl;1.p.*4.p.*5.p*.7.p) \\
=v=== (p\v{r}edlo\v{z}ka, se 4.p.*6.p.) \\
==n\v{e}m\v{z}==(z\'{a}jmn\'{e}no  vzta.,sg;6.p:fem,mus l,mus n.) \\
=ka\v{z}d=\'{y}==(z\'{a}jmeno; sg; [mus l.;1.p],[mus n.;1.p*4.p]) \\
=k=do== (z\'{a}jmeno;sg;1.p) \\
=mil=uje==(verb,3.os sg, actv, oznamovaci,) \\
=studi=um==(ssubstantivum,neo;sg;1.p.*4.p.*5.p.) \\
=filosofi==e=(substantivum,fem;[sg;1.p.*5.p],[pl;1.p.*4.p.*5.p.] )\\
=perspektiv==y=(substantivum, femin.[sg;2.p],[pl;1.p*4.p*5.p])\\
=mal=b=y=(substantivum, femin.[sg;2.p],[pl;1.p*4.p*5.p])\\ 
=architektu=r=y=(substantivum, femin.[sg;2.p],[pl;1.p*4.p*5.p])\\
=hud=db=y=(substantivum, femin.[sg;2.p],[pl;1.p*4.p*5.p])\\
=socha\v{r}stv\'{i}==(substantivum,neu;[sg,1.p.-*6.p.],[pl,1.p.*2.p.*4.p.*5.p.]) \\
=a===(spojka/ \v{c}\'{a}stice) \\
=dal\v{s}==\'{i}ch=( adjektivum,pl;2.p.*6.p.v\v{s}echny rody) \\
=matemati=ck=\'{y}ch=(adjektivum,pl;2.p.*6.p.v\v{s}echny rody) \\
=discipl\'{i}n===(sunstantivum,fem,pl;2.p.) \\
=nalez=ne==(verb,3.os; sg.;oznam., act.)
=velmi===(p\v{r}\'{i}slovce) \\
=jemn==\'{e}=(adjectivum, [fem;[sing,[2.p*3.p*6.p]],[pl,[2.p*3.p*5.p]]],[mas l;pl,4.p],[mas n;[pl,[2.p*3.p*5.p]]], [neu,[sg,[2.p*3.p*5.p]]])\\
=znamenit==\'{e}=(adjectivum, [fem;[sing,[2.p*3.p*6.p]],[pl,[2.p*3.p*5.p]]],[mas l;pl,4.p],[mas n;[pl,[2.p*3.p*5.p]]], [neu,[sg,[2.p*3.p*5.p]]])\\
=a===(spojka/ \v{c}\'{a}stice) \\
=subtiln=\'{i}==(adjectivum, [mas. l ; [sg; 1.p.*4.p], [pl;1.p*.4.p*5.p]]; \\
=pou\v{c}en\'{i}==(substantivum,neu;[sg,1.p.-*6.p.],[pl,1.p.*2.p.*4.p.*5.p.]) \\
=a===(spojka/ \v{c}\'{a}stice) \\
=tak\'{e}===(p\v{r}\'{i}slovce/\v{c}\'{a}stice/z\'{a}jm\'{e}no [fem;[sing,[2.p*3.p*6.p]],[pl,[2.p*3.p*5.p]]],[mas l;pl,4.p],[mas n;[pl,[2.p*3.p*5.p]]], [neu,[sg,[2.p*3.p*5.p]]])) \\
=s=e==(z\'{a}jm\'{e}no / p\v{r}edlo\v{z}ka) \\
=pot\v{e}\v{s}\'{i}(verb,3.os;act;oznamoc;sg,pl)
=r\r{u}znorod==\'{y}mi=(adjectivum,pl;7.p; v\v{s}echny rody) \\
=ot\'{a}z=k-ami=(substantivum,fem., pl.7.p) \\
==je\v{z}==(zajmeno vzta\v{z}n\'{e},[fem;sg;1.p.],[neut;sg;1.p.*4p.],[mus l. ;sg;4.p],[ostat.rody.;pl;1.p.*4.p.]/verb,2.os;pl;act;rozkaz)
=s=e==(zvratn. zajm/predl se 7.p) \\
=dot\'{y}k=aj=\'{i}=(verb,3.os;pl,sg:act.oznamavaci) \\
=velmi===(p\v{r}\'{i}slovce) \\
=tajemn==\'{y}ch=(adjektivum,pl;2.p.*6.p.v\v{s}echny rody) \\
=v\v{e}domost==\'{i}=(substantivum, fem;[sg,7.p],[pl,2.p.]) \\
\clearpage
\section{Cvi\v{c}en\'{i} 5 }
prvni v\v{e}ta je typ S6 Kombinace neprojektivn\'{i}ch konstrukc\'{i} v jedn\'{e} klauzuli \\
druh\'{a} v\v{e}ta je typ S2 syntagma \\
\section{Cvi\v{c}en\'{i} 7 }
1.v\v{e}ta Jemn\v{e} pohladil Frant\'{i}kovy/i vlasy. \\
2.v\v{e}ta Vid\v{e}l mu\v{z}e. \\
3.v\v{e}ta \v{C}ekal na let. \\
4.v\v{e}ta Ko\v{c}ka je \v{s}pinav\'{a} a l\'{i}n\'{a}. \\
5.v\v{e}ta B\'{y}/\'{i}t v\v{e}tv\'{i}. \\




\end{document}

