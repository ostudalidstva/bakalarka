\documentclass[12pt,a4paper]{article}
% Pri pouziti masivnejsich fontu (pripad pana Jurenky) lze uzit mensi pismo:
%\documentclass[11pt,a4paper]{article}
%
\usepackage{color} %pro barevné odkazy, pøíp. nadpisy
\definecolor{odkazy}{rgb}{0.21,0.27,0.53} %tmavì modrá
\definecolor{nadpisy}{rgb}{0.5812,0.0665,0.0659} %cihlová
%
% Volba pdf driveru:
%\usepackage[dvipdfm]{hyperref}%
%\usepackage[pdftex]{hyperref}%
% Parametry prevodu do pdf
\providecommand{\hypersetup}[1]{}%
\hypersetup{%
%unicode,% ? Pravdepodobne bezvyznamne
pdfauthor={jméno autora},
pdftitle={název práce},
pdfsubject={subject práce},
pdfkeywords={klicova slova},
pdffitwindow=false,% Inicialni umisteni textu v okne Readeru
bookmarksopen=true,% Panel zalozek inicialne zobrazen
% Je-li tohle nastaveno jinak, nektere odkazy nekdy nefunguji
hypertexnames=false,
plainpages=false,
%pdfpagelabels,
%
breaklinks=true,% Radkovy lom smi prijit do klikatelneho odkazu
linkcolor=odkazy,% Graficka podoba odkazu
citecolor=odkazy,% ...
colorlinks=true,% ...
pdfhighlight=/O% ... (vzhled odkazu pri stisknuti)
}%
% Inputenc je asi zbytecne.
% Alespon v MikTeXu je kodovani lepsi specifikovat v prikazovem radku:
%\usepackage[cp1250]{inputenc}
% Option 'split' ovlivnuje deleni slov obsahujicich v sobe rozdelovnik
\usepackage[czech]{babel} %dnes už je však hotová integrace èeštiny do babelu
%\usepackage[split]{czech} %dnes už je však hotová integrace èeštiny do babelu
%
\usepackage{logdp} %užiteèné drobnosti
\usepackage{amsthm} %lepší práce s vìtami
\usepackage{amsmath} %nová prostøedí pro matematiku a vylepšení tìch stávajících
\usepackage{latexsym,amsfonts,amssymb} %nová písmenka
\usepackage{fancyhdr} %záhlaví a zápatí
\usepackage[nottoc]{tocbibind} %pøidá do obsahu položky Literatura a Rejstøík
%\usepackage{colorsection} %chcete-li barevné nadpisy,
% (použije se barva z definice \definecolor{nadpisy})
% Makra pro sazbu dukazu v gentzenovskem kalkulu vytvoril Sam Buss
%\usepackage{bussproofs}

% Zmena radkovani je uzitecna pri volbe masivnich fontu, viz vyse
%\def\baselinestretch{1.2} %øádkování
\pagestyle{plain}
%pøedbìžné nastavení hlavièky (balík fancyhdr)
%\headheight 13.6pt %možná ji bude tøeba zvednout, fancyhdr si pak stìžuje: \headheight
% too small, make it at least Xpt
\headheight 14.5pt %možná ji bude tøeba zvednout, fancyhdr si pak stìžuje: \headheight too \fancyhead{}
\fancyhead[R]{\leftmark}
\fancyfoot{}
\fancyfoot[C]{\thepage}
%
\newtheorem{veta}{Vìta}%[subsection]
\newtheorem{lemma}{Lemma}
\newtheorem{tvrzeni}{Tvrzení}
%a neèíslované varianty
\newtheorem*{vetax}{Vìta}
\newtheorem*{lemmax}{Lemma}
\newtheorem*{tvrzenix}{Tvrzení}
\theoremstyle{definition}
\newtheorem*{definice}{Definice}
%\usepackage{makeidx}
%\makeindex %bude-li rejstøík
\begin{document}
%titulní stránka
\begin{titlepage}
%\fontsize{16.16pt}{25pt}\selectfont
\Large
\begin{center}
Univerzita Karlova v Praze, Filozofická fakulta\\
Katedra logiky

\vspace{8.5em}
\textsc{Michal Ketner}\\[1.4em]
{%\color{nadpisy} %pro pøíp. barevný název
Z\'{a}po\v{c}tov\'{a} cvi\v{c}en\'{i} a probl\'{e}my\\[6.8em]
2015}
\end{center}
\end{titlepage}\

\tableofcontents
\clearpage







\pagestyle{fancy} %detailní definice chování záhlaví
\renewcommand{\sectionmark}[1]{\markboth{\slshape\thesection.\ #1}{}}
\section{Cvi\v{c}en\'{i} 1 }
\subsection{V\v{e}ta a}
Vzpom\'{i}nka na tuto pohnutou ud\'{a}lost a osobn\'{i}  za\v{z}itek z jedine\v{c}n\'{e}ho rozhledu z vrcholku hory T\'{a}bor podn\'{i}tily patrn\v{e} obyvatele k tomu, \v{z}e dali po\v{r}\'{i}dit mapu zasl\'{i}ben\'{e}  zem\v{e} s jej\'{i}mi posv\'{a}tn\'{y}mi m\'{i}sty a z t\'{e}to ilustrace k Bibli si ud\v{e}lali zvl\'{a}\v{s}tn\'{i}  v\'{y}zdobu sv\'{e}  svatyn\v{e}.

\begin{tabular}{|c||c||c|}
\hline
Morfologick\'{a} & Syntaktick\'{a} & S\'{e}mantick\'{a}\\
\hline
\hline
Na  $ \rightarrow $ tuto & druhý & třetí \\
\hline
Na  $ \rightarrow $ pohnutou & druhý & třetí \\
\hline
Na  $ \rightarrow $ ud\'{a}lost & pé děgrám & uhusk flo z  \\
\hline
ud\'{a}lost  $ \rightarrow $ tuto & druhý & třetí \\
\hline 
ud\'{a}lost  $ \rightarrow $ pohnutou & udy vyz & ni zapust tiglidět \\
\hline
z\'{a}\v{z}itek $ \rightarrow $ osobn\'{i} & žlýr děh & Těč \\
\hline
Z  $ \rightarrow $ jedine\v{c}n\'{e}ho & druhý & třetí \\
\hline
Z  $ \rightarrow $ rozhledu & druhý & třetí \\
\hline
Rozhledu  $ \rightarrow $ jedine\v{c}n\'{e}ho & druhý & třetí \\
\hline
Z  $ \rightarrow $ vrcholku & druhý & třetí \\
\hline
vrcholku  $ \rightarrow $ hory & druhý & třetí \\
\hline
hory  $ \rightarrow $ T\'{a}bor & druhý & třetí \\
\hline
hor  $ \rightarrow $ T\'{a}bor & druhý & třetí \\
\end{tabular}


\begin{veta}[Pythagorova]\label{veta:Pythagorova}%\index{vìta!Pythagorova}
\[  a^2 + b^2 = c^2\]
\end{veta}

\begin{proof} %prostøedí pro dùkazy z balíku amsthm
Ponecháme na ètenáøi.
\end{proof}

\begin{definice}%\bfindex{Eigenvariable} %tuèná položka rejstøíku, viz balík vsehochut
Pod pojmem \emph{Eigenvariable} rozumíme volnou promìnnou, pøes níž probíhá kvantifikace pøi užití pravidla generalizace.
\end{definice}
\end{document}
